\documentclass[11pt]{article}

% ===================== PACKAGES =====================
\usepackage[a4paper,margin=1in]{geometry}
\usepackage{amsmath,amssymb}
\usepackage{enumitem}
\usepackage{fancyhdr}
\usepackage{xcolor}
\usepackage[hidelinks]{hyperref}

% ===================== HEADER & FOOTER =====================
\pagestyle{fancy}
\fancyhf{}
\lhead{CSE400: Fundamentals of Probability in Computing}
\rhead{Lecture 11 Scribe}
\cfoot{\thepage}

% ===================== TITLE INFO =====================
\title{\textbf{Lecture 11 Scribe: Transformation of Random Variables}\\
\vspace{0.15cm}\large February 10, 2026}
\author{\textbf{Niharika Ashar}\\ AU ID: \textbf{AU2440175}\\
\vspace{0.15cm}\small Instructor: Dhaval Patel, PhD (Associate Professor)\\
SEAS -- Ahmedabad University, Ahmedabad, Gujarat, India}
\date{}

\begin{document}
\maketitle
\vspace{-0.3cm}
\hrule
\vspace{0.4cm}

% ===================== 1. OUTLINE =====================
\section{Outline and Learning Objective}

\subsection{Topics in this lecture}
\begin{enumerate}[leftmargin=*]
    \item \textbf{Transformation of Random Variables:} learning transformation techniques for random variables.
    \item \textbf{Function of Two Random Variables:} joint transformations and derived distributions.
    \item \textbf{Illustrative Example:} detailed derivation for the case $Z=X+Y$.
\end{enumerate}

\subsection{Assumption / setup stated in the outline}
\textbf{Assumption:} the PDF of the original RV (e.g., $f_X(x)$) is known \emph{a priori}; the goal is how to find the PDF of the new transformed RV (e.g., $f_Y(y)$).

\subsection{Noted example transformations (as written on the outline slide)}
\[
Z_1=X+Y,\quad
Z_2=X-Y,\quad
Z_3=\frac{X}{Y},\quad
Z_4=\sqrt{X^2+Y^2}.
\]

% ===================== 2. ONE RV TRANSFORMATION =====================
\section{Transformation of One Random Variable: $Y=g(X)$}

\subsection{Definitions and notation (as used on the slides)}
A transformation is defined by
\[
Y=g(X).
\]
CDF notation: $F_Y(y)$ and $F_X(x)$. \\
PDF notation: $f_Y(y)$ and $f_X(x)$. \\
The slides proceed via \textbf{Step S1 (CDF)} then \textbf{Step S2 (differentiate w.r.t. $y$)}.

\subsection{Assumption / condition: monotonicity}
The slide distinguishes monotonic cases:
\begin{itemize}[leftmargin=*]
    \item \textbf{Monotonically increasing} case
    \item \textbf{Monotonically decreasing} case
\end{itemize}

\subsection{Case A: $g(\cdot)$ is monotonically increasing}

\subsubsection*{Step S1 (CDF method)}
\[
F_Y(y)
= \Pr(Y \le y)
= \Pr(g(X)\le y)
= \Pr\!\big(X \le g^{-1}(y)\big)
= F_X\!\big(g^{-1}(y)\big).
\]

\subsubsection*{Step S2 (Differentiate w.r.t. $y$)}
\[
f_Y(y)
= \frac{d}{dy}\Big[F_X\!\big(g^{-1}(y)\big)\Big]
= f_X\!\big(g^{-1}(y)\big)\cdot \frac{d}{dy}\big(g^{-1}(y)\big).
\]
Let $x=g^{-1}(y)$:
\[
f_Y(y)= f_X(x)\,\frac{dx}{dy}\Big|_{x=g^{-1}(y)}.
\]

\subsection{Case B: $g(\cdot)$ is monotonically decreasing}

\subsubsection*{Step S1 (CDF method)}
\[
F_Y(y)
= \Pr(Y\le y)
= \Pr\!\big(X \ge g^{-1}(y)\big)
= 1 - F_X\!\big(g^{-1}(y)\big).
\]

\subsubsection*{Step S2 (Differentiate w.r.t. $y$)}
\[
f_Y(y)
=\frac{f_X(x)}{\left|\dfrac{dy}{dx}\right|}\Bigg|_{x=g^{-1}(y)}.
\]
\textbf{S3 note:} change the limits for $y$ (determine valid $y$-range from the mapping).

% ===================== 3. WORKED EXAMPLE =====================
\section{Worked Example: $X\sim \mathrm{Uniform}(-1,1)$, \; $Y=\sin\!\left(\dfrac{\pi x}{2}\right)$}

\subsection{Given (from the example slide)}
\begin{itemize}[leftmargin=*]
    \item $X$ is uniform on $(-1,1)$.
    \item Transformation:
    \[
    Y=g(x)=\sin\!\left(\frac{\pi x}{2}\right).
    \]
    \item PDF of $X$:
    \[
    f_X(x)=
    \begin{cases}
    \dfrac{1}{2}, & -1<x<1,\\
    0, & \text{otherwise}.
    \end{cases}
    \]
    \item Objective: find $f_Y(y)$.
\end{itemize}

\subsection{Step-by-step solution (as shown)}

\subsubsection*{Step 1: Invert the transformation}
\[
y=\sin\!\left(\frac{\pi x}{2}\right)
\quad\Rightarrow\quad
x=\frac{2}{\pi}\sin^{-1}(y).
\]

\subsubsection*{Step 2: Differentiate $x$ w.r.t. $y$}
\[
\frac{dx}{dy}=\frac{2}{\pi}\cdot\frac{1}{\sqrt{1-y^2}}.
\]

\subsubsection*{Step 3: Apply the transformation formula}
\[
f_Y(y)= f_X(x)\cdot\left|\frac{dx}{dy}\right|.
\]
Substitute $f_X(x)=\frac{1}{2}$:
\[
f_Y(y)
=\frac{1}{2}\cdot\frac{2}{\pi}\cdot\frac{1}{\sqrt{1-y^2}}
= \frac{1}{\pi\sqrt{1-y^2}}.
\]

\subsubsection*{Step 4: Determine the valid range of $y$ from endpoints}
\begin{itemize}[leftmargin=*]
    \item At $x=-1$: $y=\sin\!\left(-\frac{\pi}{2}\right)=-1$
    \item At $x=1$:  $y=\sin\!\left(\frac{\pi}{2}\right)=1$
\end{itemize}
So the support is $-1<y<1$ (otherwise $0$).

\subsection{Final answer (as presented)}
\[
f_Y(y)=
\begin{cases}
\dfrac{1}{\pi\sqrt{1-y^2}}, & -1<y<1,\\[6pt]
0, & \text{otherwise}.
\end{cases}
\]

% ===================== 4. TWO RV FUNCTION =====================
\section{Function of Two Random Variables: Example Setup ($Z=X+Y$)}

\subsection{Problem statements listed on the slide}
For $Z=X+Y$, the slide lists:
\begin{enumerate}[leftmargin=*]
    \item Find the PDF of $Z$: $f_Z(z)$.
    \item Find $f_Z(z)$ if $X$ and $Y$ are independent.
    \item Let $X\sim N(0,1)$ and $Y\sim N(0,1)$; prove that $Z\sim N(0,2)$.
    \item If $X$ and $Y$ are exponential RVs with parameter $\lambda$, find $f_Z(z)$.
\end{enumerate}

% ===================== 5. CDF DERIVATION SETUP =====================
\section{Detailed Derivation: CDF setup for $Z=X+Y$ via Region Integration}

\subsection{Definition of $Z$ and CDF start}
\[
Z=X+Y,
\qquad
F_Z(z)=\Pr(Z\le z)=\Pr(X+Y\le z).
\]

\subsection{Region description and the “strip” integrals}
Using the joint pdf $f_{XY}(x,y)$ and the boundary line $x+y=z$:

\subsubsection*{(A) Horizontal strip (H-strip) form}
\[
F_Z(z)
=
\int_{-\infty}^{\infty}
\int_{-\infty}^{\,z-y}
f_{XY}(x,y)\,dx\,dy.
\]

\subsubsection*{(B) Vertical strip form}
\[
F_Z(z)
=
\int_{-\infty}^{\infty}
\int_{-\infty}^{\,z-x}
f_{XY}(x,y)\,dy\,dx.
\]

\vspace{0.3cm}
\hrule
\vspace{0.2cm}
\noindent\textit{End of Lecture 11 scribe.}

\end{document}
